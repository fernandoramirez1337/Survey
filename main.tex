\documentclass[conference,compsoc]{IEEEtran}
\usepackage{graphicx}
\usepackage{cite}
%\usepackage[latin1]{inputenc}
\usepackage[english]{babel}

\begin{document}

\title{LIGHTWEIGHT APPROACHES IN BLOCK CRYPTOGRAPHY: A SURVEY}

\author{\IEEEauthorblockN{Fernando Ramirez Arredondo}
\IEEEauthorblockA{School of Computer Science\\
Universidad Católica San Pablo\\
Email: fernando.ramirez@ucsp.edu.pe}
}

\maketitle
\begin{abstract}
Lightweight Cryptography (LWC) is an expanding field of research as a result of its importance in providing security, via encrypting sensitive information, for IoT devices and other systems where traditional cryptography would be impractical due to their resource demands. It is for this reason that several LWC primitives have been proposed. Among these, lightweight block ciphers are the most popular in terms of methods proposed and usage. These symmetric cryptographic algorithms operate on fixed-size blocks of data and transform each block independently into an output block of the same size. This work surveys lightweight block ciphers' algorithms for resource-constrained devices, which have become increasingly important in today's world for their wide range of useful applications, from convenience and efficiency in our daily routines to far-reaching applications in healthcare, industry, and urban planning. The goal of this work is to provide a comprehensible classification of these methods in terms of performance and resources required, followed by an accurate explanation of the tradeoff between these two when working with limited resources.
\end{abstract}

\begin{IEEEkeywords} Lightweight cryptography (LWC), Lightweight block ciphers, Resource-constrained devices \end{IEEEkeywords}

\section{Introduction}
Back in 2016, World Economic Forum founder and executive chairman Klaus Schwab wrote a book titled The Fourth Industrial Revolution, in which he explained that the way we live, work and relate to one another was about to be fundamentally altered by a digital revolution unlike the ones humanity had experienced before. Back to today, this so-called fourth industrial revolution has been characterized by a fusion of technologies that is blurring the lines between the physical, digital, and biological spheres\cite{WorldEconomicForum}.
The Internet of Things (IoT) paradigm corresponds to the inter-device domain created by devices embedded with electronics, software, sensors, and connectivity, enabling the transmission, receiving, and processing of data through various communication infrastructures. IoT generates more opportunities for direct integration between the physical world and computer-based systems.
As of 2019, IoT, which used to operate in smaller network spaces, has expanded to wide area networks, thereby increasing associated risks due to the expected surge in IoT devices in diverse environments. With the rapid growth of IoT applications, a substantial amount of sensitive data is generated and exchanged, leading to the observation of several security and privacy issues. As devices become more interconnected, security and privacy concerns will become more pronounced, continuously exposing additional security flaws and weaknesses. In statistical terms, all exposed errors and weaknesses may be exploited in an environment with billions of devices.
IoT devices can be categorized into two groups based on their resources. The first group includes devices abundant in resources like servers, personal computers, tablets, smartphones, etc. The second group comprises devices limited in resources, such as industrial sensors or sensor nodes, RFID tags, actuators, etc.
This last group of devices face multiple constraints, including limitations in resources, memory, power/energy consumption, and speed. The challenge in terms of performance arises when attempting to implement traditional cryptography in devices with limited resources. This difficulty is attributed to the intricate and resource-intensive mathematical operations involved in traditional cryptographic techniques. Such operations require substantial memory space and demand high processing power. Moreover, the implementation cost of traditional cryptography in low-resource devices is notably high. Consequently, lightweight cryptography was introduced to address these issues. Lightweight cryptography, born out of the remarkable expansion in emerging ubiquitous technologies, represents a modern cryptographic technique.
Ciphers are categorized as either asymmetric or symmetric. While asymmetric ciphers provide enhanced security features, they require more computational power and tend to be relatively more costly. It is for this reason that asymmetric ciphers are not popular in the context of resource-constrained devices. Block ciphers and stream ciphers are the two main classifications of symmetric ciphers. This work will focus on block ciphers because most of the research in the LWC field centers around this paradigm.
Several lightweight block ciphers have been suggested to gain performance advantages compared to traditional cryptographic methods. Certain ciphers achieved this by simplifying well-established block ciphers, enhancing their efficiency. Notable examples include AES-128\cite{AES-128}, a variant of NIST's Advanced Encryption Standard (AES), and DESL\cite{DESL}, which is a modified version of DES. In DESL, the round function and permutations were reduced to enhance hardware implementation efficiency.
On the other hand, some algorithms are dedicated block ciphers crafted from the ground up. An early example is PRESENT\cite{PRESENT}, specifically designed for resource-constrained hardware environments. Additionally, there are families of lightweight block ciphers such as SIMON and SPECK\cite{SIMONSPECK}. These were created to be uncomplicated, adaptable, and to deliver optimal performance in both hardware and software implementations.
The advantages in performance seen in lightweight block ciphers compared to traditional block ciphers result from design decisions that prioritize smaller block sizes, reduced key sizes, simpler rounds, more straightforward key schedules, and minimal implementations. These characteristics will be discussed in detail in the upcoming sections.
The rest of the paper is organized as follows: Section 2 delves into the theorical framework of lightweight block ciphers. This section aims to establish a solid foundation by discussing background information on lightweight ciphers and their significance in ensuring security in low-resource devices. Theorical concepts and principles underlying these ciphers will be thoroughly examined. Section 3 presents a survey of related work in the realm of lightweight block ciphers. This includes a comprehensive review of existing literature, comparing and contrasting various approaches. Section 4 introduces our taxonomy of the cipher implementation space. This taxonomy serves as a structured framework for categorizing and understanding different aspects of lightweight block ciphers. It explores variations, modes of operation, and algorithmic choices, providing a roadmap for researchers and practitioners to navigate this complex landscape. In Section 5, we conduct an in-depth analysis of the strengths and weaknesses inherent in lightweight block ciphers. This assessment encompasses scenarios where these ciphers excel in terms of security and performance, as well as potential vulnerabilities or trade-offs that should be considered in their application. Finally, in Section 6, the paper is concluded with a summary of key findings and overarching conclusions. This section encapsulates the contributions of our work, discusses the implications of our taxonomy, and provides closing remarks on the significance of lightweight block ciphers in contemporary cryptographic applications.

\section{Theorical framework}
\subsection{Lightweight Cryptography}
\subsubsection{Lightweight Block Ciphers} 
\subsubsection{Lightweight Hash Functions} 
\subsubsection{Lightweight Message Authentication Codes} 
\subsubsection{Lightweight Stream Ciphers} 
\subsection{Target Devices}
\subsection{Applications and Architectures}
IoT systems offer a wide range of services, including Intelligent Transportation Systems (ITS), smart grids, smart buildings, smart cities, e-Health, intelligent drug delivery systems, and more. Even Cyber-Physical Systems (CPS), such as Nuclear Power Plants (NPP), are encompassed within the IoT framework. The majority of these services are of a critical nature. Each IoT system is tailored to deliver a specific service, and for each application, diverse architectures are in development. 
Nevertheless, the shared drawbacks among many proposed architectures prevent them from fully meeting all IoT requirements. The applications can be classified into various domains, including Medical, Military, Industrial, Automobile, Environmental, Agriculture, Retail, and Consumer. Each domain offers its unique advantages while simultaneously tackling the IoT-related challenges discussed earlier.
\subsection{Performance Metrics} 
\subsubsection{Hardware-Specific Metrics} 
\subsubsection{Software-Specific Metrics} 
\subsection{Security Analysis Techniques}
\subsubsection{Linear Cryptanalysis} 
\subsubsection{Differential Cryptanalysis} 
\subsubsection{Zero Correlation Attack} 
\subsubsection{Biclique Attack} 
\subsubsection{Avalanche Effect Attack} 
\subsection{Classification of Attacks}
\subsubsection{Protocol-Based Attacks} 
\subsubsection{Data-Based Attacks} 

\section{Related Work}

\section{Taxonomy}

\section{Strengths and Weaknesses}

\section{Conclusions}

\bibliographystyle{IEEEtran}
\bibliography{../bibliography/references}
\end{document}

